% !TeX root = TeX-Talk.tex
% !TeX encoding = UTF-8
% !TeX program = XeLaTeX
\part{设计文档格式}

\section{基本原则}

\begin{frame}[fragile]{\CJKsout{格式与内容分离}不要在意细节}
\CJKsout{格式与内容分离}\alert{不要在意细节}是 \LaTeX{} 的一大“卖点”。它使得 \LaTeX{} 不仅仅是 \TeX{} 这样一种排版语言,也是一种文档编写工具。\LaTeX{} 是面向文档作者本人的排版语言。\pause

在 \LaTeX{} 的设计中,将文档的格式设计与内容分离开来。标准的 \LaTeXe{} 文档类具有相对固定的排版格式,作者编写文档只使用 |\title|、|\section|、|abstract| 这样的命令或环境,而不必考虑其具体实现。而有关格式的细节代码,则被封装在文档类、宏包中,或在导言区分离编写。\pause

出版社提供的投稿用文档类,以及清华薛瑞尼编写的 Thuthesis 模板,北大刘玙的 pkuthss 模板,就是将事先设计好的格式交给文档作者使用的结果。
\end{frame}

\begin{frame}[fragile]{使用内容相关的命令与环境}
但是,格式与内容的分离不仅需要格式设计者的努力,也需要作者在填写内容时遵循分离原则。基本的方法就是只使用与内容相关的命令和环境。\pause
\begin{itemize}[<+->]
\item
\begin{tabbing}
推荐:\= |It is \emph{important}.| \\
不好:\> |It is \textit{important}.| 
\end{tabbing}
\item
\begin{tabbing}
推荐:\= |\caption{流程图}| \\
不好:\> |\textbf{图1:} 流程图|
\end{tabbing}
\item
\begin{tabbing}
推荐:\= |\begin{verse} 诗行 \end{verse}| \\
不好:\> |\begin{center} 诗行 \end{center}| \\
糟糕:\> |~~~~~~~~~~~~~~~~~~~~~~诗行|
\end{tabbing}
\end{itemize}
\end{frame}

\section{使用宏包}

\begin{frame}{使用宏包}
\begin{description}[<+->]
\item[作用] 宏包将可重用的代码提取出来,相当于其他程序语言中的“库”。使用宏包可以用简单的接口实现非常复杂的功能,有些对于个人来说是“不可能的任务”。
\item[问题] 第三方宏包可能破坏 \TeX{} 设计的“向前兼容性”;不同宏包之间如果出现兼容性问题更难解决。——使用宏包会将兼容性问题从 \TeX{} 语言扩大到所有宏包代码。
\end{description}
\only<+->{现代 \LaTeX{} 文档离不开第三方宏包,但应合理使用:}
\begin{itemize}[<+->]
\item 尽量不造轮子
\item 尽量排除不需要的宏包
\end{itemize}
\end{frame}


\section{格式控制功能}

\begin{frame}[fragile]{字体字号}
字体
\begin{itemize}
\item |\rmfamily|, |\textrm{...}|
\item |\sffamily|, |\textsf{...}|
\item |\ttfamily|, |\texttt{...}|
\end{itemize}

字号:|\Huge|, |\LARGE|, |\Large|, |\large|, |\normalsize|, |\small|, |\footnotesize|, |\scriptsize|, |\tiny|

中文字号:|\zihao{5}|、|\zihao{-3}|
\end{frame}

\begin{frame}[fragile]{对齐}
|\centering|、|\raggedleft|、|\raggedright|
\end{frame}

\begin{frame}[fragile]{空白间距}
|\hspace{2cm}|

|\vspace{3mm}|
\end{frame}

\begin{frame}{版面布局}
\pkg{geometry} 宏包

\pkg{fancyhdr} 宏包等
\end{frame}

\begin{frame}[fragile]{分页断行}
|\linebreak|、|\\|

|\pagebreak|、|\newpage|、|\clearpage|、|\cleardoublepage|
\end{frame}

\begin{frame}[fragile]{盒子}
|\mbox{内容}|

|\parbox{4em}{内容}|、|minipage|
\end{frame}

\section{格式应用于文档}

\begin{frame}[fragile]{使用在导言区单独设置格式}
如果预定义的格式不符合需要,就需要设置修改。经常文档作者本人就是格式设计者,此时更应该注意不要把格式和内容混在一起。\pause
\begin{itemize}[<+->]
\item 直接设置相关参数。如设置 |\parindent|、|\parskip|、|\linespread|、|\pagestyle|。
\item 修改部分命令定义。如修改 |\thesection|、|\labelenumi|、|\descriptionlabel|、|\figurename|。
\item 利用工具宏包完成设置。如使用 \pkg{ctex} 宏包设置中文格式,使用 \pkg{tocloft} 宏包设置目录格式。
\end{itemize}\pause
传统的文档中经常修改 \LaTeX{} 的内部命令,如重定义内部命令 |\l@section| 来修改目录格式。这体现了当初 \LaTeX{} 设计的不足:没有提供足够的用户层接口来调整格式。不过这类方法比较晦涩,一般应该使用相关宏包功能代替。
\end{frame}

\begin{frame}[fragile]{利用自定义命令和环境}
对于 \LaTeX{} 没有直接提供的格式,可以使用自定义的命令和环境实现语义的接口。\pause

例如,为程序名称定义一个命令:
\begin{Verbatim}
\newcommand\prg[1]{\textsf{#1}}
\end{Verbatim}
\pause
这不仅提供了更具意义的名字,而且为未来的修改和扩充提供条件:
\begin{Verbatim}
\newcommand\prg[1]{%
  \textcolor{blue}\texttt{#1}\index{#1 程序}}
\end{Verbatim}
\pause

\alert{注意:}各种直接修改输出格式的命令,如字体、字号、对齐、间距的命令,都应该放在文档格式设置或自定义的命令、环境中,而避免在正文中直接使用。
\end{frame}

\begin{frame}[fragile]{章节标题}
\pkg{ctex} 宏包及文档类,用 |\ctexset| 定制。西文用 \pkg{titlesec} 等。
\begin{Verbatim}
\ctexset {
  chapter = {
    beforeskip = 0pt,
    fixskip = true,
    format = \Huge\bfseries,
    nameformat = \rule{\linewidth}{1bp}\par
                 \bigskip\hfill\chapternamebox,
    number = \arabic{chapter},
    aftername = \par\medskip,
    aftertitle = \par\bigskip\nointerlineskip
                 \rule{\linewidth}{2bp}\par}}
\newcommand\chapternamebox[1]{%
  \parbox{\ccwd}{\linespread{1}\selectfont\centering #1}}
\end{Verbatim}
\end{frame}

\begin{frame}{浮动标题}
\pkg{caption} 宏包
\end{frame}

\begin{frame}{列表环境}
\pkg{enumitem} 宏包
\end{frame}
