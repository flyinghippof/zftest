% !TeX root = TeX-Talk.tex
% !TeX encoding = UTF-8
% !TeX program = XeLaTeX
\part{组织文档结构}

\section{编写结构化文档}

\begin{frame}{文档由什么组成?}
\begin{itemize}
\item 标题
\item 前言/摘要
\item 目录
\item 正文
\begin{itemize}
\item 篇、章、节、小节、小段
\begin{itemize}
\item 文字、公式
\item 列表:编号的、不编号的、带小标题的
\item 定理、引理、命题、证明、结论
\item 诗歌、引文、程序代码、算法伪码
\item 制表
\item 画图
\end{itemize}
\end{itemize}
\item 文献
\item 索引、词汇表
\end{itemize}
\end{frame}

\begin{frame}{纲举目张}{编写结构化文档}
\LaTeX{} 支持结构化的文档编写方式,也只有具有良好结构的文档才适合使用 \LaTeX{} 来编写。\pause

步骤:
\begin{itemize}
\item 拟定主题
\item 列出提纲
\item 填写内容
\item \CJKsout{调整格式}\alert{不要在意格式}
\end{itemize}
\end{frame}

\begin{frame}{Markdown:最简单的结构标记语言}
\begin{itemize}
\item 各级标题:对应于文章章节

\item 两种列表:编号、不编号

\item 强调文字:弱、强

\item 插入代码:行内代码、大段代码

\item 插图与链接

\item 一些扩展(如数学公式)
\end{itemize}
\end{frame}

\begin{frame}{Markdown 演示}
纯文本

Typora
\end{frame}

\begin{frame}{\LyX{}:结构化的文档写作系统}
\LyX{} 是一个图形界面的接近“所见即所得”效果的文档处理软件。\LyX{} 可以模拟 \LaTeX{} 的大部分功能,也可以生成 \LaTeX{} 代码。

\LyX{} 不是 \LaTeX{} 编辑器,它不能编辑任意的 \LaTeX{} 文档代码。
\end{frame}

\begin{frame}{\LaTeX{}:结构化文档语言}
可以用任何文本编辑器编写,可以使用专门的编辑器(如 TeXworks)或通用的代码编辑器(如 VS code)。
\end{frame}

\section{\LaTeX{}:结构化文档语言}

\begin{frame}[fragile]{\LaTeX{} 文档基本结构}
以 |document| 环境为界,|document| 环境前是导言部分(preamble);环境内部是正文部分;环境之后的部分被忽略。

在导言区进行格式设置,正文部分套用格式。\pause

\begin{Verbatim}
%%% 简单文档
% 导言:格式设置
\documentclass{ctexart}
\usepackage[b5paper]{geometry}
% 正文:填写内容
\begin{document}
使用 \LaTeX
\end{document}
\end{Verbatim}

\end{frame}


\begin{frame}[fragile]{文档部件}
\begin{itemize}
\item 标题:|\title|, |\author|, |\date| —— |\maketitle|
\item 摘要/前言:|abstract| 环境 / |\chapter*|
\item 目录:|\tableofcontents|
\item 章节:|\chapter|, |\section|,\ldots
\item 附录:|\appendix| $+$ |\chapter|或|\section| \ldots
\item 文献:|\bibliography|
\item 索引:|\printindex|
\end{itemize}
\end{frame}

\begin{frame}[fragile]{文档划分}
大型文档:|\frontmatter|、|\mainmatter|、|\backmatter|\pause

一般文档:|\appendix|
\pause

\begin{table}
\small
\begin{tabular}{rlll}
\toprule
层次 & 名称                    & 命令             & 说明 \\
\midrule
$-1$ & part            & |\part|          & 可选的最高层 \\
0    & chapter           & |\chapter|       & \pkg{report},
\pkg{book} 类最高层 \\
1    & section           & |\section|       & \pkg{article} 类最高层 \\
2    & subsection      & |\subsection|    & \\
3    & subsubsection & |\subsubsection| &
\makecell[tl]{\pkg{report}, \pkg{book} 类 \\ 默认不编号、不编目录} \\
4    & paragraph         & |\paragraph|     & 默认不编号、不编目录 \\
5    & subparagraph    & |\subparagraph|  & 默认不编号、不编目录 \\
\bottomrule
\end{tabular}
\caption{章节层次}\label{tab:sectioning}
\end{table}
\pause
\footnotesize
|secnumdepth| 编号的深度,|tocdepth| 编目的深度。默认值均为 3。
\end{frame}

\begin{frame}[fragile]{磁盘文件组织}
小文档将所有内容写在同一个目录中。对比较大的文档,可以将文档分成多个文件,并划分文件目录结构:
\begin{itemize}
\item 主文档,给出文档框架结构
\item 按内容章节划分不同的文件
\item 使用单独的类文件和格式文件设置格式
\item 用小文件隔离复杂的图表
\end{itemize}\pause
相关命令:
\begin{itemize}
\item |\documentclass|:读入文档类文件(|.cls|)
\item |\usepackage|:读入一个格式文件——宏包(|.sty|)
\item |\include|:分页,并读入章节文件(|.tex|)
\item |\input|:读入任意的文件
\end{itemize}
\end{frame}

\begin{frame}[shrink=20]{文档框架示例}
\begin{multicols}{2}
\VerbatimInput{sample/skeleton.tex}
\end{multicols}
\end{frame}
