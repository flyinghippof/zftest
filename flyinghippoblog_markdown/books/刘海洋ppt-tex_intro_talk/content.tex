% !TeX root = TeX-Talk.tex
% !TeX encoding = UTF-8
% !TeX program = XeLaTeX
\part{填写文档内容}

\section{\LaTeX{} 基础}

\begin{frame}[fragile]{迟到的 Hello world.}
\begin{columns}
\column{0.5\textwidth}
找个东西输入文本:
\begin{Verbatim}
\documentclass{article}
\begin{document}
Hello world.
\end{document}
\end{Verbatim}
\pause
编译代码得到结果:
\begin{quote}
Hello world.
\end{quote}
\pause
\column{0.5\textwidth}
中文几乎没有改变:
\begin{Verbatim}
\documentclass{ctexart}
\begin{document}
今天你吃了吗?
\end{document}
\end{Verbatim}
得到:
\begin{quote}
今天你吃了吗?
\end{quote}
\end{columns}
\pause
具体如何编译:

\begin{tikzpicture}[auto,>=latex,shorten >=1pt]
\scope[every node/.style={draw,fill=blue!10}]
\node (src) at (0,0) {.tex 文件};
\node (pdf) at (6,0) {.pdf 文件};
\node[draw=gray,text=gray] (dvi) at (1,-1.5) {.dvi 文件};
\node[draw=gray,text=gray] (ps) at (5,-1.5) {.ps 文件};
\endscope
\scope[every node/.style={font={\scriptsize\ttfamily},inner sep=0pt}]
\draw[->] (src) edge[bend left=25] node{pdflatex(英文推荐)} (pdf);
\draw[->,thick] (src) edge[bend left=4] node{xelatex(中文推荐)} (pdf);
\draw[->] (src) edge[bend right=4] node[swap]{lualatex(未广泛使用)} (pdf);
\draw[->,gray] (src) -- node[swap]{latex} (dvi);
\draw[->,gray] (dvi) -- node[swap]{dvipdfmx} (pdf);
\draw[->,gray] (dvi) -- node[swap]{dvips(旧)} (ps);
\draw[->,gray] (ps) -- node[swap]{ps2pdf} (pdf);
\endscope
\node[anchor=west,text width=3.5cm] at (7,0)
    {为了生成目录、引用信息,往往需要若干次编译};
\end{tikzpicture}
\end{frame}

\begin{frame}[fragile]{语法结构}

相比原始的 \TeX{} 语言,\LaTeX{} 的语法结构被限制为相对固定的形式。\pause

\begin{itemize}[<+->]
\item 命令:参数总在后面花括号表示,用中括号表示可选参数
\begin{Verbatim}
\cmd{arg1}{arg2}\\
\cmd[opt]{arg1}{arg2}
\end{Verbatim}
\begin{tabbing}
\LaTeX{} 的分数 $\frac12$ \= |\frac{1}{2}|\\
\TeX{} 的分数 $\frac12$ \> |1 \over 2|
\end{tabbing}

\item 环境
\begin{Verbatim}
\begin{env}
  ......
\end{env}
\end{Verbatim}
\begin{tabbing}
\LaTeX{} 的矩阵 \= |\begin{matrix} ... \\ ... \end{matrix}|\\
\TeX{} 的矩阵 \> |\matrix{...\cr ...\cr}|
\end{tabbing}

\item 注释:以符号 |%| 开头,该行在 |%| 后面的部分。
\end{itemize}
\end{frame}

\begin{frame}[fragile]{\LaTeX{} 宏:命令与环境}
\LaTeX{} 中的宏可分为命令与环境:\pause
\begin{description}
\item[命令]
命令通常以反斜线开头,可以带零到多个参数。命令也可以是直接输出某种结果;也可以改变一个状态,此时 \LaTeX{} 用花括号 |{}| 分组或环境作为状态改变的作用域。

例如 |\em abc| 改变字体以强调一些文字,得到 {\em abc};而带参数的命令 |\emph{abc}| 可得到同样的效果。
\pause
\item[环境]
环境的格式为
\begin{Verbatim}
\begin{env}
  环境的内容
\end{env}
\end{Verbatim}
例如右对齐:
\begin{columns}
\column{4cm}
\begin{Verbatim}
\begin{flushright}
文字
\end{flushright}
\end{Verbatim}
\column{4cm}
\begin{quote}
\begin{flushright}
文字
\end{flushright}
\end{quote}
\end{columns}
\end{description}
\end{frame}


\section{正文文本}

\begin{frame}{正文文本}
直接输入正文文本。

用空格分开单词。一个换行符等同于一个空格,多个空格的效果与一个相同。

自然段分段是空一行。
\end{frame}

\begin{frame}[fragile]{正文符号}
一些符号被 \LaTeX{} 宏语言所占用,需要以命令形式输入:
\begin{democode}
\# \$ \% \& \{ \}
\textbackslash
\end{democode}

键盘上没有的符号用命令输入。
\begin{democode}
\S \dag \ddag \P \copyright
\textbullet \textregistered
\texttrademark \pounds
\end{democode}

更多的符号需要使用符号字体包。(看 symbols 文档)
\end{frame}

\section{公式}

\subsection{数学公式}

\begin{frame}[fragile]{数学模式}
数学模式下字体、符号、间距与正文都不同,一切数学公式(包括单个符号 $n$, $\pi$)都要在数学模式下输入。
\begin{itemize}
\item 行内(inline)公式:使用一对符号 |$| |$| 来标示。如 |$a+b=c$|。
\item 显示(display)公式。
\begin{itemize}
\item 简单的不编号公式用命令 |\[| 和 |\]| 标示。(不要使用双美元符号 |$$| |$$|)
\item 基本的编号的公式用 |equation| 环境。
\item 更复杂的结构,使用 \pkg{amsmath} 宏包提供的专门的数学环境。(不要使用 |eqnarray| 环境)
\end{itemize}
\end{itemize}
\end{frame}

\begin{frame}[fragile]{数学结构}
\begin{itemize}
\item 上标与下标:用 |^| 和 |_| 表示。
\item 上下画线与花括号:|\overline|, |\underline|, |\overbrace|, |\underbrace|
\item 分式:|\frac{分子}{分母}|
\item 根式:|\sqrt[次数]{根号下}|
\item 矩阵:使用 \pkg{amsmath} 宏包提供的专门的矩阵环境 |matrix|, |pmatrix|, |bmatrix| 等。特别复杂的矩阵(如带线条)使用 |array| 环境作为表格画出。
\end{itemize}
\end{frame}

\begin{frame}[fragile]{数学符号}
\begin{itemize}
\item 数学字母 $a$, $b$, $\alpha$, $\Delta$,数学字体 |\mathbb|($\mathbb{R}$)、|\mathcal|($\mathcal{P}$)等
\item 普通符号:如 |\infty|($\infty$), |\angle|($\angle$)
\item 二元运算符:$a+b$, $a-b$ 及 $a\oplus b$
\item 二元关系符:$a=b$, $a\le b$
\item 括号:$\langle a, b\rangle$,使用 |\left|, |\right| 放大
\item 标点:逗号、分号(|\colon|)
\end{itemize}
\end{frame}


\begin{frame}[fragile]{\pkg{amsmath} 与 \pkg{mathtools}}
\pkg{amsmath} 是基本的数学工具包,在包含数学公式的文档中几乎无处不在。\pkg{mathtools} 则对 \pkg{amsmath} 做了一些补充和增强。\pause

例子:
\begin{align*}
2^5 &= (1+1)^5 \\
   &= \begin{multlined}[t]
      \binom50\cdot 1^5 + \binom51\cdot 1^4 \cdot 1
        + \binom52\cdot 1^3 \cdot 1^2 \\
      + \binom53\cdot 1^2 \cdot 1^3 + \binom54\cdot 1 \cdot 1^4
        + \binom55\cdot 1^5
    \end{multlined} \\
   &= \binom50 + \binom51 + \binom52 + \binom53
        + \binom54 + \binom55
\end{align*}
\end{frame}

\begin{frame}[fragile]{示例代码}
\begin{Verbatim}
\begin{align*}
2^5 &= (1+1)^5 \\
    &= \begin{multlined}[t]
      \binom50\cdot 1^5 + \binom51\cdot 1^4 \cdot 1
        + \binom52\cdot 1^3 \cdot 1^2 \\
      + \binom53\cdot 1^2 \cdot 1^3
        + \binom54\cdot 1 \cdot 1^4 + \binom55\cdot 1^5
    \end{multlined} \\
    &= \binom50 + \binom51 + \binom52 + \binom53
        + \binom54 + \binom55
\end{align*}
\end{Verbatim}
\end{frame}

\subsection{科技功能}

\begin{frame}[fragile]{\pkg{siunitx}:数字单位的一揽子解决方案}
\begin{democode}
\num{-1.235e96} \\
\SI{299792458}{m/s} \\
\SI{2x7x3.5}{m}
\end{democode}
\pause
\begin{democode}
\begin{tabular}{|S|}\hline
-234532\\ 13.55 \\ .9e37km \\
\hline
\end{tabular}
\end{democode}
\pause
注:\pkg{siunitx} 的代码有整个 \LaTeX{} 内核那么长。
\end{frame}

\begin{frame}[fragile]{\pkg{chemformula}:编写化学式}
\pkg{chemformula} 宏包(过去用 \pkg{mhchem})是在 \TeX{} 中定义新语法规则的典范。它让化学反应式的书写比数学式还要容易,绝大部分功能只需要 |\ch| 一条命令:
\begin{democode}
\ch{2 H2 + O2 -> 2 H2O}\\
\ch{2H2O -> 2 H2 ^ + O2 ^}
\end{democode}
\end{frame}


\section{列表与文本块}

\begin{frame}[fragile]{列表环境}
\begin{itemize}
\item |enumerate| 编号
\item |itemize| 不编号
\item |description| 有标题
\end{itemize}
\end{frame}

\begin{frame}[fragile]{定理类环境}
\begin{itemize}
\item |\newtheorem| 定义定理类环境,如
\begin{Verbatim}
\newtheorem{thm}{定理}[section]
\end{Verbatim}
\item 使用定理类环境,如
\begin{democode}
\begin{thm}
一个定理
\end{thm}
\end{democode}
\end{itemize}
\end{frame}

\begin{frame}[fragile]{诗歌与引文}
\begin{itemize}
\item |verse|
\item |quote|
\item |quotation|
\end{itemize}
\end{frame}

\begin{frame}[fragile]{抄录代码}
\begin{itemize}
\item |\verb| 命令,如
\begin{democode}
代码 \verb|#include <stdio.h>|
\end{democode}
\item |verbatim|
\begin{democode}
\begin{verbatim}
#include <stdio.h>
int main() {
    puts("hello world.");
}
\end{verbatim}
\end{democode}
\end{itemize}
\end{frame}

\begin{frame}[fragile]{高级代码:语法高亮}
\begin{itemize}
\item 使用 \pkg{listings} 宏包
\begin{democode}
\begin{lstlisting}[language=C,
  basicstyle=\ttfamily,
  stringstyle=\color{blue}]
#include <stdio.h>
int main() {
  puts("hello world.");
}
\end{lstlisting}
\end{democode}
\item \pkg{minted} 宏包(调用 Pygment)
\end{itemize}
\end{frame}

\begin{frame}{算法结构}
\begin{itemize}
\item \pkg{clrscode} 宏包(算法导论)
\item \pkg{algorithm2e} 宏包
\item \pkg{algorithmicx} 宏包的 \pkg{algpseudocode} 格式
\end{itemize}
\end{frame}

\begin{frame}[fragile,shrink=10]{算法结构:\pkg{clrscode} 示例}
\begin{vdemocode}
% \usepackage{clrscode}
\begin{codebox}
\Procname{$\proc{Merge-Sort}(A,p,r)$}
\li \If $p<r$
\li \Then $q \gets \lfloor(p+r)/2\rfloor$
\li   $\proc{Merge-Sort}(A,p,q)$
\li   $\proc{Merge-Sort}(A,q+1,r)$
\li   $\proc{Merge}(A,p,q,r)$
    \End
\end{codebox}
\end{vdemocode}
\end{frame}

\section{图表与浮动环境}

\begin{frame}[fragile]{画表格}
使用 |tabular| 环境。
\begin{democode}
\begin{tabular}{|rr|}
\hline
输入& 输出\\ \hline
$-2$ & 4 \\
0 & 0 \\
2 & 4 \\ \hline
\end{tabular}
\end{democode}

可以使用一些工具生成表格代码,例如\\
\url{https://www.tablesgenerator.com/latex_tables}
\end{frame}

\begin{frame}{功能各异的表格宏包}
\begin{itemize}
\item 单元格处理:\pkg{multirow}、\pkg{makecell}
\item 长表格:\pkg{longtable}、\pkg{xtab}
\item 定宽表格:\pkg{xtabular}
\item 表线控制:\pkg{booktabs}、\pkg{diagbox}、\pkg{arydshln}
\item 表列格式:\pkg{array}
\item 综合应用:\pkg{tabu}
\end{itemize}
\end{frame}

\begin{frame}[fragile]{插图}
使用 \pkg{graphicx} 宏包提供的 |\includegraphics| 命令。
\begin{vdemocode}
\includegraphics[width=2cm]{pkulogo.pdf}
\end{vdemocode}
\end{frame}

\begin{frame}{代码画图}
优先使用外部工具画图,特别是可视化工具,例如一般的矢量图用 Inkscape、Illustrator 甚至 PowerPoint(保存为 pdf 格式),数学图形用 MATLAB、matplotlib 之类。

如果有合适的宏包,某些特定类型的图形也可以用 \LaTeX{} 代码作图。现代 \LaTeX{} 绘图宏包很多基于 \pkg{TikZ}。
\end{frame}

\begin{frame}[fragile]{浮动体}
\begin{itemize}
\item |figure| 环境
\item |table| 环境
\item 其他环境可以使用 \pkg{float} 宏包得到
\end{itemize}

浮动体的标题用 |\caption| 命令得到,自动编号。
\end{frame}


\section{自动化工具}

\subsection{目录与引用}

\begin{frame}[fragile]{目录}
\begin{figure}
\centering
\begin{tikzpicture}[doc/.style={draw,fill=shade,shape=rectangle},
  prog/.style={draw,fill=shade,shape=ellipse},
  lab/.style={auto,font=\small},
  >=latex,shorten >=2pt]
  \coordinate (in);
  \node[doc,right=of in] (tex) {|.tex| 源文档};
  \node[prog,right=of tex] (latex) {\LaTeX{} 引擎};
  \node[doc,right=of latex] (pdf) {PDF/DVI 文件};
  \coordinate[right=of pdf] (out);
  \node[doc,above=of latex] (toc) {|.toc| 目录文件};
  \draw[->] (in) -- node[lab]{编写} (tex);
  \draw[->] (tex) -- node[lab]{输入} (latex);
  \draw[->] (latex) -- node[lab]{输出} (pdf);
  \draw[->] (pdf) -- node[lab] {发布} (out);
  \draw[->,dashed] (latex.120) -- (latex.120 |- toc.south)
    node[midway,lab,left]{前一次编译};
  \draw[->,dashed] (latex.60 |- toc.south) -- (latex.60)
    node[midway,lab,right]{再次编译};
\end{tikzpicture}
\caption{\LaTeX{} 章节目录生成示意图}
\label{fig:toc}
\end{figure}
\end{frame}

\begin{frame}[fragile]{交叉引用工作原理}
\begin{figure}
\centering
\begin{tikzpicture}[doc/.style={draw,fill=shade,shape=rectangle},
  prog/.style={draw,fill=shade,shape=ellipse},
  lab/.style={auto,font=\small},
  >=latex,shorten >=2pt]
  \coordinate (in);
  \node[doc,right=of in] (tex) {|.tex| 源文档};
  \node[prog,right=of tex] (latex) {\LaTeX{} 引擎};
  \node[doc,right=of latex] (pdf) {PDF/DVI 文件};
  \coordinate[right=of pdf] (out);
  \node[doc,above=of latex] (aux) {|.aux| 辅助文件};
  \draw[->] (in) -- node[lab]{编写} (tex);
  \draw[->] (tex) -- node[lab]{输入} (latex);
  \draw[->] (latex) -- node[lab]{输出} (pdf);
  \draw[->] (pdf) -- node[lab] {发布} (out);
  \draw[->,dashed] (latex.120) -- (latex.120 |- aux.south)
    node[midway,lab,left]{前一次编译};
  \draw[->,dashed] (latex.60 |- aux.south) -- (latex.60)
    node[midway,lab,right]{再次编译};
  \node[right=1 of aux,ellipse callout,inner sep=0,
    callout pointer shorten=2pt,callout absolute pointer=(aux.east),
    draw,font={\small\itshape}]
    {\shortstack{标签、编号、\\页码……}};
\end{tikzpicture}
\caption{\LaTeX{} 交叉引用生成示意图}
\label{fig:xref}
\end{figure}
\end{frame}

\begin{frame}[fragile]{\pkg{hyperref}:PDF 的链接与书签}
\pkg{hyperref} 产生链接和书签的原理与普通的交叉引用相同。\pkg{hyperref} 会在 PDF 中写入相应的“锚点”代码,在其他地方引用。交叉引用的代码并入 |.aux| 文件,目录的代码并入 |.toc| 文件,PDF 书签则产生单独的 |.out| 文件。
\end{frame}


\subsection{\BibTeX}

\begin{frame}[fragile]{\BibTeX{} 工作原理}
\begin{figure}
\begin{lrbox}{\tmpbox}
\begin{tikzpicture}[doc/.style={draw,fill=shade,shape=rectangle},
  prog/.style={draw,fill=shade,shape=ellipse},
  lab/.style={auto,swap,midway,font=\small},
  >=latex,shorten >=2pt]
  \matrix[column sep=1cm,row sep=1cm] {
    && \node[prog] (xe1) {|xelatex|};  & \node[doc] (pdf1) {无文献 PDF}; & \\
    && \node[doc] (aux1) {|.aux| 辅助}; & \node[prog] (btx) {|bibtex|};  & \node[doc] (bbl) {|.bbl| 文献列表}; \\
  \node (in) {编写}; & \node[doc] (tex) {|.tex| 源文件}; & \node[prog] (xe2) {|xelatex|}; & \node[doc] (pdf2) {无引用 PDF}; & \\
    && \node[doc] (aux2) {|.aux| 辅助}; & &  \\
    && \node[prog] (xe3) {|xelatex|}; & \node[doc] (pdf3) {最终 PDF}; & \node (out) {发布}; \\
  };
  \node[above left=0.4 and 0 of btx,doc] (bib) {|.bib| 数据库};
  \node[above right=0.4 and 0 of btx,doc] (bst) {|.bst| 格式};
  \scope[->]
  \scope[very thick]
  \draw (in) -- (tex);
  \draw (tex) to[bend left] node[lab] {1} (xe1);
  \draw (tex) -- (xe2) node[lab] {3};
  \draw (tex) to[bend right] node[lab,swap] {4} (xe3);
  \draw (xe1) -- (pdf1) node[lab,swap] {1};
  \draw (xe2) -- (pdf2) node[lab] {3};
  \draw (xe3) -- (pdf3) node[lab] {4};
  \draw (pdf3) -- (out);
  \endscope
  \draw (bib) -- (btx) node[lab] {2};
  \draw (bst) -- (btx) node[lab,swap] {2};
  \draw (aux1) -- (btx) node[lab] {2};
  \draw (btx) -- (bbl) node[lab] {2};
  \scope[dashed]
  \draw (xe1) -- (aux1) node[lab] {1};
  \draw (aux1) -- (xe2) node[lab] {3};
  \draw (xe2) -- (aux2) node[lab] {3};
  \draw (aux2) -- (xe3) node[lab] {4};
  \draw (bbl) -- (xe2) node[lab] {3};
  \draw (bbl) to[bend left] node[lab] {4} (xe3);
  \endscope
  \endscope
\end{tikzpicture}
\end{lrbox}
\resizebox{.9\textwidth}{!}{\usebox{\tmpbox}}
\caption{\BibTeX{} 编译处理流程。这里以 \XeLaTeX{} 为例。}
\label{fig:bibtex}
\end{figure}
\end{frame}

\begin{frame}[fragile]{设置文献格式}
\begin{itemize}[<+->]
\item 选用合适的 |.bst| 格式,比如 |plainnat|,|gbt7714-plain|。
\item \pkg{natbib} 与作者-年格式
\item 利用 \pkg{custom-bib} 产生定制的格式文件
\item \pkg{biblatex} + Biber:文献处理的新方式
\end{itemize}
\end{frame}
